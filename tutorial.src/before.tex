Before you start, please read these points carefully:

\begin{itemize}
\item
  First of all, you must have a recent version ($3.0.0$ or higher) of R
  installed. You can download the latest version of R from this page:
  \url{http://cran.r-project.org/}.
\item
  The lavaan package is not finished yet. But it is already very useful
  for most users, or so we hope. However, some important features that
  are currently \emph{NOT} available in lavaan are:

  \begin{itemize}
  \item
    support for hierarchical/multilevel datasets (multilevel cfa,
    multilevel sem)
  \item
    support for discrete latent variables (mixture models, latent
    classes)
  \item
    Bayesian estimation
  \end{itemize}

  We hope to add these features in the next (two?) year(s) or so.
\item
  We consider the current version as \emph{beta} software. This does NOT
  mean that you can not trust the results. We believe the results are
  accurate. It does mean that things may change when new versions come
  out. For example, we may change the name of the arguments in function
  calls. And we change the internals of the source code constantly.
  However, the model syntax is fairly mature and has been stable for a
  while.
\item
  We do not expect you to be an expert in R. In fact, the lavaan package
  is designed to be used by users that would normally never use R.
  Nevertheless, it may help to familiarize yourself a bit with R, just
  to be comfortable with it. Perhaps the most important skill that you
  may need to learn is how to import your own datasets (perhaps in an
  SPSS format) into R. There are many tutorials on the web to teach you
  just that. Once you have your data in R, you can start specifying your
  model. We have tried very hard to make it as easy as possible for
  users to fit their models. Of course, if you have suggestions on how
  we can improve things, please let us know.
\item
  This document is written for first-time users of the lavaan package.
  It is not a reference manual, nor does it contain technical material
  on how things are done in the lavaan package. These documents are
  currently under preparation.
\item
  The lavaan package is free open-source software. This means (among
  other things) that there is no warranty whatsoever.
\item
  If you need help, you can ask questions in the lavaan discussion
  group. Go to \url{https://groups.google.com/d/forum/lavaan/} and join
  the group. Once you have joined the group, you can email your
  questions to
  \href{mailto:lavaan@googlegroups.com}{lavaan@googlegroups.com}. If you
  think you have found a bug, or if you have a suggestion for
  improvement, you can either email me directly (to alert me), post it
  to the discussion group (to discuss it), or open an issue on github
  (see \url{https://github.com/yrosseel/lavaan/issues}). The latter is
  useful once we have agreed it is a bug, and it should be fixed. If you
  report a bug, it is always very useful to provide a reproducible
  example (a short R script and some data).
\end{itemize}
