Consider a classical mediation setup with three variables: Y is the
dependent variable, X is the predictor, and M is a mediator. For
illustration, we create a toy dataset containing these three variables,
and fit a path analysis model that includes the direct effect of X on Y
and the indirect effect of X on Y via M.

\begin{Shaded}
\begin{Highlighting}[]
\KeywordTok{set.seed}\NormalTok{(}\DecValTok{1234}\NormalTok{)}
\NormalTok{X <-}\StringTok{ }\KeywordTok{rnorm}\NormalTok{(}\DecValTok{100}\NormalTok{)}
\NormalTok{M <-}\StringTok{ }\FloatTok{0.5}\NormalTok{*X +}\StringTok{ }\KeywordTok{rnorm}\NormalTok{(}\DecValTok{100}\NormalTok{)}
\NormalTok{Y <-}\StringTok{ }\FloatTok{0.7}\NormalTok{*M +}\StringTok{ }\KeywordTok{rnorm}\NormalTok{(}\DecValTok{100}\NormalTok{)}
\NormalTok{Data <-}\StringTok{ }\KeywordTok{data.frame}\NormalTok{(}\DataTypeTok{X =} \NormalTok{X, }\DataTypeTok{Y =} \NormalTok{Y, }\DataTypeTok{M =} \NormalTok{M)}
\NormalTok{model <-}\StringTok{ ' # direct effect}
\StringTok{             Y ~ c*X}
\StringTok{           # mediator}
\StringTok{             M ~ a*X}
\StringTok{             Y ~ b*M}
\StringTok{           # indirect effect (a*b)}
\StringTok{             ab := a*b}
\StringTok{           # total effect}
\StringTok{             total := c + (a*b)}
\StringTok{         '}
\NormalTok{fit <-}\StringTok{ }\KeywordTok{sem}\NormalTok{(model, }\DataTypeTok{data =} \NormalTok{Data)}
\KeywordTok{summary}\NormalTok{(fit)}
\end{Highlighting}
\end{Shaded}

\begin{verbatim}
lavaan (0.5-13) converged normally after  13 iterations

  Number of observations                           100

  Estimator                                         ML
  Minimum Function Test Statistic                0.000
  Degrees of freedom                                 0
  P-value (Chi-square)                           0.000

Parameter estimates:

  Information                                 Expected
  Standard Errors                             Standard

                   Estimate  Std.err  Z-value  P(>|z|)
Regressions:
  Y ~
    X         (c)     0.036    0.104    0.348    0.728
  M ~
    X         (a)     0.474    0.103    4.613    0.000
  Y ~
    M         (b)     0.788    0.092    8.539    0.000

Variances:
    Y                 0.898    0.127
    M                 1.054    0.149

Defined parameters:
    ab                0.374    0.092    4.059    0.000
    total             0.410    0.125    3.287    0.001
\end{verbatim}

The example illustrates the use of the \texttt{":="} operator in the
lavaan model syntax. This operator `defines' new parameters which take
on values that are an arbitrary function of the original model
parameters. The function, however, must be specified in terms of the
parameter \emph{labels} that are explicitly mentioned in the model
syntax. By default, the standard errors for these defined parameters are
computed by using the so-called Delta method. As with other models,
bootstrap standard errors can be requested simply by specifying
\texttt{se = "bootstrap"} in the fitting function.
