The lavaan package has full support for multiple groups. To request a
multiple group analysis, you need to add the name of the group variable
in your dataset to the argument \texttt{group} in the fitting function.
By default, the same model is fitted in all groups. In the following
example, we fit the H\&S CFA model for the two schools (Pasteur and
Grant-White).

\begin{Shaded}
\begin{Highlighting}[]
\NormalTok{HS.model <-}\StringTok{ '  visual =~ x1 + x2 + x3}
\StringTok{              textual =~ x4 + x5 + x6}
\StringTok{              speed   =~ x7 + x8 + x9 '}
\NormalTok{fit <-}\StringTok{ }\KeywordTok{cfa}\NormalTok{(HS.model, }
           \DataTypeTok{data =} \NormalTok{HolzingerSwineford1939, }
           \DataTypeTok{group =} \StringTok{"school"}\NormalTok{)}
\KeywordTok{summary}\NormalTok{(fit)}
\end{Highlighting}
\end{Shaded}

\begin{verbatim}
lavaan (0.5-13) converged normally after  63 iterations

  Number of observations per group         
  Pasteur                                          156
  Grant-White                                      145

  Estimator                                         ML
  Minimum Function Test Statistic              115.851
  Degrees of freedom                                48
  P-value (Chi-square)                           0.000

Chi-square for each group:

  Pasteur                                       64.309
  Grant-White                                   51.542

Parameter estimates:

  Information                                 Expected
  Standard Errors                             Standard

Group 1 [Pasteur]:

                   Estimate  Std.err  Z-value  P(>|z|)
Latent variables:
  visual =~
    x1                1.000
    x2                0.394    0.122    3.220    0.001
    x3                0.570    0.140    4.076    0.000
  textual =~
    x4                1.000
    x5                1.183    0.102   11.613    0.000
    x6                0.875    0.077   11.421    0.000
  speed =~
    x7                1.000
    x8                1.125    0.277    4.057    0.000
    x9                0.922    0.225    4.104    0.000

Covariances:
  visual ~~
    textual           0.479    0.106    4.531    0.000
    speed             0.185    0.077    2.397    0.017
  textual ~~
    speed             0.182    0.069    2.628    0.009

Intercepts:
    x1                4.941    0.095   52.249    0.000
    x2                5.984    0.098   60.949    0.000
    x3                2.487    0.093   26.778    0.000
    x4                2.823    0.092   30.689    0.000
    x5                3.995    0.105   38.183    0.000
    x6                1.922    0.079   24.321    0.000
    x7                4.432    0.087   51.181    0.000
    x8                5.563    0.078   71.214    0.000
    x9                5.418    0.079   68.440    0.000
    visual            0.000
    textual           0.000
    speed             0.000

Variances:
    x1                0.298    0.232
    x2                1.334    0.158
    x3                0.989    0.136
    x4                0.425    0.069
    x5                0.456    0.086
    x6                0.290    0.050
    x7                0.820    0.125
    x8                0.510    0.116
    x9                0.680    0.104
    visual            1.097    0.276
    textual           0.894    0.150
    speed             0.350    0.126



Group 2 [Grant-White]:

                   Estimate  Std.err  Z-value  P(>|z|)
Latent variables:
  visual =~
    x1                1.000
    x2                0.736    0.155    4.760    0.000
    x3                0.925    0.166    5.583    0.000
  textual =~
    x4                1.000
    x5                0.990    0.087   11.418    0.000
    x6                0.963    0.085   11.377    0.000
  speed =~
    x7                1.000
    x8                1.226    0.187    6.569    0.000
    x9                1.058    0.165    6.429    0.000

Covariances:
  visual ~~
    textual           0.408    0.098    4.153    0.000
    speed             0.276    0.076    3.639    0.000
  textual ~~
    speed             0.222    0.073    3.022    0.003

Intercepts:
    x1                4.930    0.095   51.696    0.000
    x2                6.200    0.092   67.416    0.000
    x3                1.996    0.086   23.195    0.000
    x4                3.317    0.093   35.625    0.000
    x5                4.712    0.096   48.986    0.000
    x6                2.469    0.094   26.277    0.000
    x7                3.921    0.086   45.819    0.000
    x8                5.488    0.087   63.174    0.000
    x9                5.327    0.085   62.571    0.000
    visual            0.000
    textual           0.000
    speed             0.000

Variances:
    x1                0.715    0.126
    x2                0.899    0.123
    x3                0.557    0.103
    x4                0.315    0.065
    x5                0.419    0.072
    x6                0.406    0.069
    x7                0.600    0.091
    x8                0.401    0.094
    x9                0.535    0.089
    visual            0.604    0.160
    textual           0.942    0.152
    speed             0.461    0.118
\end{verbatim}

If you want to fix parameters, or provide starting values, you can use
the same pre-multiplication techniques, but the single argument is now
replaced by a \emph{vector} of arguments, one for each group. If you use
a single element instead of a vector, that element will be applied for
all groups (note: this is NOT true for labels, since this would imply
equality constraints). For example:

\begin{Shaded}
\begin{Highlighting}[]
\NormalTok{HS.model <-}\StringTok{ '  visual =~ x1 + 0.5*x2 + c(0.6, 0.8)*x3}
\StringTok{              textual =~ x4 + start(c(1.2, 0.6))*x5 + a*x6}
\StringTok{              speed   =~ x7 + x8 + x9 '}
\end{Highlighting}
\end{Shaded}

In the definition of the latent factor \texttt{visual}, we have fixed
the factor loading of the indicator \texttt{x3} to the value `0.6' in
the first group, and to the value `0.8' in the second group, while the
factor loading of the indicator is fixed to the value `0.5' in both
groups. In the definition of the \texttt{textual} factor, two different
starting values are provided for the \texttt{x5} indicator; one for each
group. In addition, we have labeled the factor loading of the
\texttt{x6} indicator as `a', but this label is only given to the
parameter of the first group. If you want to provide labels to each of
the two groups, you can write something like \texttt{c(a1,a2)*x6}. Be
careful: if you write \texttt{c(a,a)*x6}, both parameters (in the first
and second group) will get the same label, and hence they will be
treated as a single parameter. To verify the effects of these modifiers,
we refit the model:

\begin{Shaded}
\begin{Highlighting}[]
\NormalTok{fit <-}\StringTok{ }\KeywordTok{cfa}\NormalTok{(HS.model, }
           \DataTypeTok{data =} \NormalTok{HolzingerSwineford1939, }
           \DataTypeTok{group =} \StringTok{"school"}\NormalTok{)}
 \KeywordTok{summary}\NormalTok{(fit)}
\end{Highlighting}
\end{Shaded}

\begin{verbatim}
lavaan (0.5-13) converged normally after  58 iterations

  Number of observations per group         
  Pasteur                                          156
  Grant-White                                      145

  Estimator                                         ML
  Minimum Function Test Statistic              118.976
  Degrees of freedom                                52
  P-value (Chi-square)                           0.000

Chi-square for each group:

  Pasteur                                       64.901
  Grant-White                                   54.075

Parameter estimates:

  Information                                 Expected
  Standard Errors                             Standard

Group 1 [Pasteur]:

                   Estimate  Std.err  Z-value  P(>|z|)
Latent variables:
  visual =~
    x1                1.000
    x2                0.500
    x3                0.600
  textual =~
    x4                1.000
    x5                1.185    0.102   11.598    0.000
    x6        (a)     0.876    0.077   11.409    0.000
  speed =~
    x7                1.000
    x8                1.129    0.279    4.055    0.000
    x9                0.931    0.227    4.103    0.000

Covariances:
  visual ~~
    textual           0.460    0.103    4.479    0.000
    speed             0.182    0.076    2.408    0.016
  textual ~~
    speed             0.181    0.069    2.625    0.009

Intercepts:
    x1                4.941    0.094   52.379    0.000
    x2                5.984    0.100   59.945    0.000
    x3                2.487    0.092   26.983    0.000
    x4                2.823    0.092   30.689    0.000
    x5                3.995    0.105   38.183    0.000
    x6                1.922    0.079   24.321    0.000
    x7                4.432    0.087   51.181    0.000
    x8                5.563    0.078   71.214    0.000
    x9                5.418    0.079   68.440    0.000
    visual            0.000
    textual           0.000
    speed             0.000

Variances:
    x1                0.388    0.129
    x2                1.304    0.155
    x3                0.965    0.120
    x4                0.427    0.069
    x5                0.454    0.086
    x6                0.289    0.050
    x7                0.824    0.124
    x8                0.510    0.116
    x9                0.677    0.105
    visual            1.001    0.172
    textual           0.892    0.150
    speed             0.346    0.125



Group 2 [Grant-White]:

                   Estimate  Std.err  Z-value  P(>|z|)
Latent variables:
  visual =~
    x1                1.000
    x2                0.500
    x3                0.800
  textual =~
    x4                1.000
    x5                0.990    0.087   11.425    0.000
    x6                0.963    0.085   11.374    0.000
  speed =~
    x7                1.000
    x8                1.228    0.188    6.539    0.000
    x9                1.081    0.168    6.417    0.000

Covariances:
  visual ~~
    textual           0.454    0.099    4.585    0.000
    speed             0.315    0.079    4.004    0.000
  textual ~~
    speed             0.222    0.073    3.049    0.002

Intercepts:
    x1                4.930    0.097   50.688    0.000
    x2                6.200    0.089   69.616    0.000
    x3                1.996    0.086   23.223    0.000
    x4                3.317    0.093   35.625    0.000
    x5                4.712    0.096   48.986    0.000
    x6                2.469    0.094   26.277    0.000
    x7                3.921    0.086   45.819    0.000
    x8                5.488    0.087   63.174    0.000
    x9                5.327    0.085   62.571    0.000
    visual            0.000
    textual           0.000
    speed             0.000

Variances:
    x1                0.637    0.115
    x2                0.966    0.120
    x3                0.601    0.091
    x4                0.316    0.065
    x5                0.418    0.072
    x6                0.407    0.069
    x7                0.609    0.091
    x8                0.411    0.094
    x9                0.522    0.089
    visual            0.735    0.132
    textual           0.942    0.152
    speed             0.453    0.117
\end{verbatim}

\paragraph{Constraining a single parameter to be equal across groups}

If you want to constrain one or more parameters to be equal across
groups, you need to give them the same label. For example, to constrain
the factor loading of the indicator \texttt{x3} to be equal across (two)
groups, you can write:

\begin{Shaded}
\begin{Highlighting}[]
\NormalTok{HS.model <-}\StringTok{ '  visual =~ x1 + x2 + c(v3,v3)*x3}
\StringTok{              textual =~ x4 + x5 + x6}
\StringTok{              speed   =~ x7 + x8 + x9 '}
\end{Highlighting}
\end{Shaded}

Again, identical labels imply identical parameters, both within and
across groups.

\paragraph{Constraining groups of parameters to be equal across groups}

Although providing identical labels is a very flexible method to specify
equality constraints for a few parameters, there is a more convenient
way to impose equality constraints on a whole set of parameters (for
example: all factor loadings, or all intercepts). We call these type of
constraints \emph{group equality constraints} and they can be specified
by the argument \texttt{group.equal} in the fitting function. For
example, to constrain (all) the factor loadings to be equal across
groups, you can proceed as follows:

\begin{Shaded}
\begin{Highlighting}[]
\NormalTok{HS.model <-}\StringTok{ '  visual =~ x1 + x2 + x3}
\StringTok{              textual =~ x4 + x5 + x6}
\StringTok{              speed   =~ x7 + x8 + x9 '}
\NormalTok{fit <-}\StringTok{ }\KeywordTok{cfa}\NormalTok{(HS.model, }
           \DataTypeTok{data =} \NormalTok{HolzingerSwineford1939, }
           \DataTypeTok{group =} \StringTok{"school"}\NormalTok{,}
           \DataTypeTok{group.equal =} \KeywordTok{c}\NormalTok{(}\StringTok{"loadings"}\NormalTok{))}
\KeywordTok{summary}\NormalTok{(fit)}
\end{Highlighting}
\end{Shaded}

\begin{verbatim}
lavaan (0.5-13) converged normally after  46 iterations

  Number of observations per group         
  Pasteur                                          156
  Grant-White                                      145

  Estimator                                         ML
  Minimum Function Test Statistic              124.044
  Degrees of freedom                                54
  P-value (Chi-square)                           0.000

Chi-square for each group:

  Pasteur                                       68.825
  Grant-White                                   55.219

Parameter estimates:

  Information                                 Expected
  Standard Errors                             Standard

Group 1 [Pasteur]:

                   Estimate  Std.err  Z-value  P(>|z|)
Latent variables:
  visual =~
    x1                1.000
    x2                0.599    0.100    5.979    0.000
    x3                0.784    0.108    7.267    0.000
  textual =~
    x4                1.000
    x5                1.083    0.067   16.049    0.000
    x6                0.912    0.058   15.785    0.000
  speed =~
    x7                1.000
    x8                1.201    0.155    7.738    0.000
    x9                1.038    0.136    7.629    0.000

Covariances:
  visual ~~
    textual           0.416    0.097    4.271    0.000
    speed             0.169    0.064    2.643    0.008
  textual ~~
    speed             0.176    0.061    2.882    0.004

Intercepts:
    x1                4.941    0.093   52.991    0.000
    x2                5.984    0.100   60.096    0.000
    x3                2.487    0.094   26.465    0.000
    x4                2.823    0.093   30.371    0.000
    x5                3.995    0.101   39.714    0.000
    x6                1.922    0.081   23.711    0.000
    x7                4.432    0.086   51.540    0.000
    x8                5.563    0.078   71.087    0.000
    x9                5.418    0.079   68.153    0.000
    visual            0.000
    textual           0.000
    speed             0.000

Variances:
    x1                0.551    0.137
    x2                1.258    0.155
    x3                0.882    0.128
    x4                0.434    0.070
    x5                0.508    0.082
    x6                0.266    0.050
    x7                0.849    0.114
    x8                0.515    0.095
    x9                0.658    0.096
    visual            0.805    0.171
    textual           0.913    0.137
    speed             0.305    0.078



Group 2 [Grant-White]:

                   Estimate  Std.err  Z-value  P(>|z|)
Latent variables:
  visual =~
    x1                1.000
    x2                0.599    0.100    5.979    0.000
    x3                0.784    0.108    7.267    0.000
  textual =~
    x4                1.000
    x5                1.083    0.067   16.049    0.000
    x6                0.912    0.058   15.785    0.000
  speed =~
    x7                1.000
    x8                1.201    0.155    7.738    0.000
    x9                1.038    0.136    7.629    0.000

Covariances:
  visual ~~
    textual           0.437    0.099    4.423    0.000
    speed             0.314    0.079    3.958    0.000
  textual ~~
    speed             0.226    0.072    3.144    0.002

Intercepts:
    x1                4.930    0.097   50.763    0.000
    x2                6.200    0.091   68.379    0.000
    x3                1.996    0.085   23.455    0.000
    x4                3.317    0.092   35.950    0.000
    x5                4.712    0.100   47.173    0.000
    x6                2.469    0.091   27.248    0.000
    x7                3.921    0.086   45.555    0.000
    x8                5.488    0.087   63.257    0.000
    x9                5.327    0.085   62.786    0.000
    visual            0.000
    textual           0.000
    speed             0.000

Variances:
    x1                0.645    0.127
    x2                0.933    0.121
    x3                0.605    0.096
    x4                0.329    0.062
    x5                0.384    0.073
    x6                0.437    0.067
    x7                0.599    0.090
    x8                0.406    0.089
    x9                0.532    0.086
    visual            0.722    0.161
    textual           0.906    0.136
    speed             0.475    0.109
\end{verbatim}

More `group equality constraints' can be added. In addition to the
factor loadings, the following keywords are supported in the
\texttt{group.equal} argument:

\begin{itemize}
\itemsep1pt\parskip0pt\parsep0pt
\item
  \texttt{intercepts}: the intercepts of the observed variables
\item
  \texttt{means}: the intercepts/means of the latent variables
\item
  \texttt{residuals}: the residual variances of the observed variables
\item
  \texttt{residual.covariances}: the residual covariances of the
  observed variables
\item
  \texttt{lv.variances}: the (residual) variances of the latent
  variables
\item
  \texttt{lv.covariances}: the (residual) covariances of the latent
  varibles
\item
  \texttt{regressions}: all regression coefficients in the model
\end{itemize}

If you omit the \texttt{group.equal} argument, all parameters are freely
estimated in each group (but the model structure is the same).

But what if you want to constrain a whole group of parameters (say all
factor loadings and intercepts) across groups, except for one or two
parameters that need to stay free in all groups. For this scenario, you
can use the argument \texttt{group.partial}, containing the names of
those parameters that need to remain free. For example:

\begin{Shaded}
\begin{Highlighting}[]
\NormalTok{fit <-}\StringTok{ }\KeywordTok{cfa}\NormalTok{(HS.model, }
           \DataTypeTok{data =} \NormalTok{HolzingerSwineford1939, }
           \DataTypeTok{group =} \StringTok{"school"}\NormalTok{,}
           \DataTypeTok{group.equal =} \KeywordTok{c}\NormalTok{(}\StringTok{"loadings"}\NormalTok{, }\StringTok{"intercepts"}\NormalTok{),}
           \DataTypeTok{group.partial =} \KeywordTok{c}\NormalTok{(}\StringTok{"visual=~x2"}\NormalTok{, }\StringTok{"x7~1"}\NormalTok{))}
\end{Highlighting}
\end{Shaded}

\paragraph{Measurement Invariance}

If you are interested in testing the measurement invariance of a CFA
model across several groups, you can use the function
\texttt{measurementInvariance()} which performs a number of multiple
group analyses in a particular sequence, with increasingly more
restrictions on the parameters. (Note: from the 0.5 series onwards, the
\texttt{measurementInvariance()} function has been moved to the
\texttt{semTools} package.) Each model is compared to the baseline model
and the previous model using chi-square difference tests. In addition,
the difference in the fit measure is also shown. Although the current
implementation of the function is still a bit primitive, it does
illustrate how the various components of the lavaan package can be used
as building blocks for constructing higher level functions (such as the
\texttt{measurementInvariance()} function), something that is often very
hard to accomplish with commercial software.

\begin{Shaded}
\begin{Highlighting}[]
\KeywordTok{library}\NormalTok{(semTools)}
\KeywordTok{measurementInvariance}\NormalTok{(HS.model, }\DataTypeTok{data =} \NormalTok{HolzingerSwineford1939, }\DataTypeTok{group =} \StringTok{"school"}\NormalTok{)}
\end{Highlighting}
\end{Shaded}

\begin{verbatim}

Measurement invariance tests:

Model 1: configural invariance:
   chisq       df   pvalue      cfi    rmsea      bic 
 115.851   48.000    0.000    0.923    0.097 7706.822 

Model 2: weak invariance (equal loadings):
   chisq       df   pvalue      cfi    rmsea      bic 
 124.044   54.000    0.000    0.921    0.093 7680.771 

[Model 1 versus model 2]
  delta.chisq      delta.df delta.p.value     delta.cfi 
        8.192         6.000         0.224         0.002 

Model 3: strong invariance (equal loadings + intercepts):
   chisq       df   pvalue      cfi    rmsea      bic 
 164.103   60.000    0.000    0.882    0.107 7686.588 

[Model 1 versus model 3]
  delta.chisq      delta.df delta.p.value     delta.cfi 
       48.251        12.000         0.000         0.041 

[Model 2 versus model 3]
  delta.chisq      delta.df delta.p.value     delta.cfi 
       40.059         6.000         0.000         0.038 

Model 4: equal loadings + intercepts + means:
   chisq       df   pvalue      cfi    rmsea      bic 
 204.605   63.000    0.000    0.840    0.122 7709.969 

[Model 1 versus model 4]
  delta.chisq      delta.df delta.p.value     delta.cfi 
       88.754        15.000         0.000         0.083 

[Model 3 versus model 4]
  delta.chisq      delta.df delta.p.value     delta.cfi 
       40.502         3.000         0.000         0.042 
\end{verbatim}

By adding the \texttt{group.partial} argument, you can test for partial
measurement invariance by allowing a few parameters to remain free.
