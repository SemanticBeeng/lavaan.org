If you have no full dataset, but you do have a sample covariance matrix,
you can still fit your model. If you wish to add a mean structure, you
need to provide a mean vector too. Importantly, if only sample
statistics are provided, you must specify the number of observations
that were used to compute the sample moments. The following example
illustrates the use of a sample covariance matrix as input. First, we
read in the lower half of the covariance matrix (including the
diagonal):

\begin{Shaded}
\begin{Highlighting}[]
\NormalTok{lower <-}\StringTok{ '}
\StringTok{ 11.834}
\StringTok{  6.947   9.364}
\StringTok{  6.819   5.091  12.532}
\StringTok{  4.783   5.028   7.495   9.986}
\StringTok{ -3.839  -3.889  -3.841  -3.625  9.610}
\StringTok{-21.899 -18.831 -21.748 -18.775 35.522 450.288 '}

\NormalTok{wheaton.cov <-}\StringTok{ }
\StringTok{    }\KeywordTok{getCov}\NormalTok{(lower, }\DataTypeTok{names =} \KeywordTok{c}\NormalTok{(}\StringTok{"anomia67"}\NormalTok{, }\StringTok{"powerless67"}\NormalTok{, }
                            \StringTok{"anomia71"}\NormalTok{, }\StringTok{"powerless71"}\NormalTok{,}
                            \StringTok{"education"}\NormalTok{, }\StringTok{"sei"}\NormalTok{))}
\end{Highlighting}
\end{Shaded}

The \texttt{getCov()} function makes it easy to create a full covariance
matrix (including variable names) if you only have the lower-half
elements (perhaps pasted from a textbook or a paper). Note that the
lower-half elements are written between two single quotes. Therefore,
you have some additional flexibility. You can add comments, and blank
lines. If the numbers are separated by a comma, or a semi-colon, that is
fine too. For more information about \texttt{getCov()}, see the online
manual page.

Next, we can specify our model, estimate it, and request a summary of
the results:

\begin{Shaded}
\begin{Highlighting}[]
\CommentTok{# classic wheaton et al model}
\NormalTok{wheaton.model <-}\StringTok{ '}
\StringTok{  # latent variables}
\StringTok{    ses     =~ education + sei}
\StringTok{    alien67 =~ anomia67 + powerless67}
\StringTok{    alien71 =~ anomia71 + powerless71}
\StringTok{  # regressions}
\StringTok{    alien71 ~ alien67 + ses}
\StringTok{    alien67 ~ ses}
\StringTok{  # correlated residuals}
\StringTok{    anomia67 ~~ anomia71}
\StringTok{    powerless67 ~~ powerless71}
\StringTok{'}
\NormalTok{fit <-}\StringTok{ }\KeywordTok{sem}\NormalTok{(wheaton.model, }
           \DataTypeTok{sample.cov =} \NormalTok{wheaton.cov, }
           \DataTypeTok{sample.nobs =} \DecValTok{932}\NormalTok{)}
\KeywordTok{summary}\NormalTok{(fit, }\DataTypeTok{standardized =} \OtherTok{TRUE}\NormalTok{)}
\end{Highlighting}
\end{Shaded}

\begin{verbatim}
lavaan (0.5-13) converged normally after  82 iterations

  Number of observations                           932

  Estimator                                         ML
  Minimum Function Test Statistic                4.735
  Degrees of freedom                                 4
  P-value (Chi-square)                           0.316

Parameter estimates:

  Information                                 Expected
  Standard Errors                             Standard

                   Estimate  Std.err  Z-value  P(>|z|)   Std.lv  Std.all
Latent variables:
  ses =~
    education         1.000                               2.607    0.842
    sei               5.219    0.422   12.364    0.000   13.609    0.642
  alien67 =~
    anomia67          1.000                               2.663    0.774
    powerless67       0.979    0.062   15.895    0.000    2.606    0.852
  alien71 =~
    anomia71          1.000                               2.850    0.805
    powerless71       0.922    0.059   15.498    0.000    2.628    0.832

Regressions:
  alien71 ~
    alien67           0.607    0.051   11.898    0.000    0.567    0.567
    ses              -0.227    0.052   -4.334    0.000   -0.207   -0.207
  alien67 ~
    ses              -0.575    0.056  -10.195    0.000   -0.563   -0.563

Covariances:
  anomia67 ~~
    anomia71          1.623    0.314    5.176    0.000    1.623    0.356
  powerless67 ~~
    powerless71       0.339    0.261    1.298    0.194    0.339    0.121

Variances:
    education         2.801    0.507                      2.801    0.292
    sei             264.597   18.126                    264.597    0.588
    anomia67          4.731    0.453                      4.731    0.400
    powerless67       2.563    0.403                      2.563    0.274
    anomia71          4.399    0.515                      4.399    0.351
    powerless71       3.070    0.434                      3.070    0.308
    ses               6.798    0.649                      1.000    1.000
    alien67           4.841    0.467                      0.683    0.683
    alien71           4.083    0.404                      0.503    0.503
\end{verbatim}

If you have multiple groups, the \texttt{sample.cov} argument must be a
list containing the sample variance-covariance matrix of each group as a
separate element in the list. If a mean structure is needed, the
\texttt{sample.mean} argument must be a list containing the sample means
of each group. Finally, the \texttt{sample.nobs} argument can be either
a list or an integer vector containing the number of observations for
each group.
