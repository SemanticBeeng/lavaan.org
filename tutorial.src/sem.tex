In our second example, we will use the built-in
\texttt{PoliticalDemocracy} dataset. This is a dataset that has been
used by Bollen in his 1989 book on structural equation modeling (and
elsewhere). To learn more about the dataset, see its help page and the
references therein.

The figure below contains a graphical representation of the model that
we want to fit.

The corresponding lavaan syntax for specifying this model is as follows:

\begin{Shaded}
\begin{Highlighting}[]
\NormalTok{model <-}\StringTok{ '}
\StringTok{  # measurement model}
\StringTok{    ind60 =~ x1 + x2 + x3}
\StringTok{    dem60 =~ y1 + y2 + y3 + y4}
\StringTok{    dem65 =~ y5 + y6 + y7 + y8}
\StringTok{  # regressions}
\StringTok{    dem60 ~ ind60}
\StringTok{    dem65 ~ ind60 + dem60}
\StringTok{  # residual correlations}
\StringTok{    y1 ~~ y5}
\StringTok{    y2 ~~ y4 + y6}
\StringTok{    y3 ~~ y7}
\StringTok{    y4 ~~ y8}
\StringTok{    y6 ~~ y8}
\StringTok{'}
\end{Highlighting}
\end{Shaded}

In this example, we use three different formula types: latent variabele
definitions (using the \texttt{=\textasciitilde{}} operator), regression
formulas (using the \texttt{\textasciitilde{}} operator), and
(co)variance formulas (using the
\texttt{\textasciitilde{}\textasciitilde{}} operator). The regression
formulas are similar to ordinary formulas in R. The (co)variance
formulas typically have the following form:

\begin{verbatim}
variable ~~ variable
\end{verbatim}

The variables can be either observed or latent variables. If the two
variable names are the same, the expression refers to the variance (or
residual variance) of that variable. If the two variable names are
different, the expression refers to the (residual) covariance among
these two variables. The lavaan package automatically makes the
distinction between variances and residual variances.

In our example, the expression
\texttt{y1 \textasciitilde{}\textasciitilde{} y5} allows the residual
variances of the two observed variables to be correlated. This is
sometimes done if it is believed that the two variables have something
in common that is not captured by the latent variables. In this case,
the two variables refer to identical scores, but measured in two
different years (1960 and 1965, respectively). Note that the two
expressions \texttt{y2 \textasciitilde{}\textasciitilde{} y4} and
\texttt{y2 \textasciitilde{}\textasciitilde{} y6}, can be combined into
the expression \texttt{y2 \textasciitilde{}\textasciitilde{}~y4 + y6}.
This is just a shorthand notation.

We enter the model syntax as follows:

\begin{Shaded}
\begin{Highlighting}[]
\NormalTok{model <-}\StringTok{ '}
\StringTok{  # measurement model}
\StringTok{    ind60 =~ x1 + x2 + x3}
\StringTok{    dem60 =~ y1 + y2 + y3 + y4}
\StringTok{    dem65 =~ y5 + y6 + y7 + y8}
\StringTok{  # regressions}
\StringTok{    dem60 ~ ind60}
\StringTok{    dem65 ~ ind60 + dem60}
\StringTok{  # residual correlations}
\StringTok{    y1 ~~ y5}
\StringTok{    y2 ~~ y4 + y6}
\StringTok{    y3 ~~ y7}
\StringTok{    y4 ~~ y8}
\StringTok{    y6 ~~ y8}
\StringTok{'}
\end{Highlighting}
\end{Shaded}

To fit the model and see the results we can type:

\begin{Shaded}
\begin{Highlighting}[]
\NormalTok{fit <-}\StringTok{ }\KeywordTok{sem}\NormalTok{(model, }\DataTypeTok{data =} \NormalTok{PoliticalDemocracy)}
\KeywordTok{summary}\NormalTok{(fit, }\DataTypeTok{standardized =} \OtherTok{TRUE}\NormalTok{)}
\end{Highlighting}
\end{Shaded}

\begin{verbatim}
lavaan (0.5-13) converged normally after  68 iterations

  Number of observations                            75

  Estimator                                         ML
  Minimum Function Test Statistic               38.125
  Degrees of freedom                                35
  P-value (Chi-square)                           0.329

Parameter estimates:

  Information                                 Expected
  Standard Errors                             Standard

                   Estimate  Std.err  Z-value  P(>|z|)   Std.lv  Std.all
Latent variables:
  ind60 =~
    x1                1.000                               0.670    0.920
    x2                2.180    0.139   15.742    0.000    1.460    0.973
    x3                1.819    0.152   11.967    0.000    1.218    0.872
  dem60 =~
    y1                1.000                               2.223    0.850
    y2                1.257    0.182    6.889    0.000    2.794    0.717
    y3                1.058    0.151    6.987    0.000    2.351    0.722
    y4                1.265    0.145    8.722    0.000    2.812    0.846
  dem65 =~
    y5                1.000                               2.103    0.808
    y6                1.186    0.169    7.024    0.000    2.493    0.746
    y7                1.280    0.160    8.002    0.000    2.691    0.824
    y8                1.266    0.158    8.007    0.000    2.662    0.828

Regressions:
  dem60 ~
    ind60             1.483    0.399    3.715    0.000    0.447    0.447
  dem65 ~
    ind60             0.572    0.221    2.586    0.010    0.182    0.182
    dem60             0.837    0.098    8.514    0.000    0.885    0.885

Covariances:
  y1 ~~
    y5                0.624    0.358    1.741    0.082    0.624    0.296
  y2 ~~
    y4                1.313    0.702    1.871    0.061    1.313    0.273
    y6                2.153    0.734    2.934    0.003    2.153    0.356
  y3 ~~
    y7                0.795    0.608    1.308    0.191    0.795    0.191
  y4 ~~
    y8                0.348    0.442    0.787    0.431    0.348    0.109
  y6 ~~
    y8                1.356    0.568    2.386    0.017    1.356    0.338

Variances:
    x1                0.082    0.019                      0.082    0.154
    x2                0.120    0.070                      0.120    0.053
    x3                0.467    0.090                      0.467    0.239
    y1                1.891    0.444                      1.891    0.277
    y2                7.373    1.374                      7.373    0.486
    y3                5.067    0.952                      5.067    0.478
    y4                3.148    0.739                      3.148    0.285
    y5                2.351    0.480                      2.351    0.347
    y6                4.954    0.914                      4.954    0.443
    y7                3.431    0.713                      3.431    0.322
    y8                3.254    0.695                      3.254    0.315
    ind60             0.448    0.087                      1.000    1.000
    dem60             3.956    0.921                      0.800    0.800
    dem65             0.172    0.215                      0.039    0.039
\end{verbatim}

The function \texttt{sem()} is very similar to the function
\texttt{cfa()}. In fact, the two functions are currently almost
identical, but this may change in the future. In the \texttt{summary()}
function, we omitted the \texttt{fit.measures=TRUE} argument. Therefore,
you only get the basic chi-square test statistic. The argument
\texttt{standardized=TRUE} augments the output with standardized
parameter values. Two extra columns of standardized parameter values are
printed. In the first column (labeled \texttt{Std.lv}), only the latent
variables are standardized. In the second column (labeled
\texttt{Std.all}), both latent and observed variables are standardized.
The latter is often called the `completely standardized solution'.

The complete code to specify and fit this model is printed again below:

\begin{Shaded}
\begin{Highlighting}[]
\KeywordTok{library}\NormalTok{(lavaan) }\CommentTok{# only needed once per session}
\NormalTok{model <-}\StringTok{ '}
\StringTok{  # measurement model}
\StringTok{    ind60 =~ x1 + x2 + x3}
\StringTok{    dem60 =~ y1 + y2 + y3 + y4}
\StringTok{    dem65 =~ y5 + y6 + y7 + y8}
\StringTok{  # regressions}
\StringTok{    dem60 ~ ind60}
\StringTok{    dem65 ~ ind60 + dem60}
\StringTok{  # residual correlations}
\StringTok{    y1 ~~ y5}
\StringTok{    y2 ~~ y4 + y6}
\StringTok{    y3 ~~ y7}
\StringTok{    y4 ~~ y8}
\StringTok{    y6 ~~ y8}
\StringTok{'}
\NormalTok{fit <-}\StringTok{ }\KeywordTok{sem}\NormalTok{(model, }\DataTypeTok{data=}\NormalTok{PoliticalDemocracy)}
\KeywordTok{summary}\NormalTok{(fit, }\DataTypeTok{standardized=}\OtherTok{TRUE}\NormalTok{)}
\end{Highlighting}
\end{Shaded}

